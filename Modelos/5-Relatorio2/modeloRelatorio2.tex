\documentclass[12pt,a4paper]{article}

\usepackage{times}
\usepackage{lipsum} %para gerar alguns textos como exemplo

\usepackage{amssymb} %para inserir fórmulas matemáticas
\usepackage{amsmath} %para inserir fórmulas matemáticas
\usepackage{amsfonts} %para inserir fórmulas matemáticas

\usepackage[brazil,brazilian]{babel}
\usepackage[T1]{fontenc} 
\usepackage[utf8]{inputenc}

\usepackage{color,graphicx} %permite a inserção de figuras
\usepackage{lineno} %usamos este pacote para retirar o espaço em branco ao redor das figuras

\usepackage{setspace} %espaçamento entre linhas (doublespacing, singlespacing)
\usepackage[top=3cm,left=3cm,right=2cm,bottom=2cm]{geometry} %definir margens


%%O pacote abaixo serve para inserir códigos
\usepackage{listings}% http://ctan.org/pkg/listings
\lstset{
  basicstyle=\ttfamily,
  mathescape
}

\providecommand{\keywords}[1]{\textbf{\textit{Keywords---}} #1}
\providecommand{\palavraschave}[1]{\textbf{\textit{Palavras-chave---}} #1}
\usepackage{multicol}
\setlength{\columnsep}{1cm}


%\usepackage[alf]{abntex2cite}%para citações alfa-numéricas padrão ABNT
%\citeoption{abnt-full-initials=yes}

\bibliographystyle{plain}%Choose a bibliograhpic style


%%%%%%%%%%%%%%%%%%%%%%%%%%%%%%%%%%%%%%%%%%%%%%%%%%%%%%%%%%%%%%%%%%%%%%%%
%%%%%%%%%%%%%%%%%%%%%%%%%%%%%%%%%%%%%%%%%%%%%%%%%%%%%%%%%%%%%%%%%%%%%%%%

%Abertura

%Introdução do título.
\title{TRABALHO PRÁTICO III}

%Introdução do autor.

\author{Fulano de Tal\footnote{alguma coisa sobre o autor 1},  Ciclano Beltrano\footnote{alguma coisa sobre o autor2} \\
\footnotesize Métodos Numéricos Computacionais, Departamento de Engenharia, CEFET-MG Nepomuceno\\
\footnotesize \texttt{\{fulano, ciclano\}@cefetmg.br} \\ }


%Definição da data
\date{\today}

%%%%%%%%%%%%%%%%%%%%%%%%%%%%%%%%%%%%%%%%%%%%%%%%%%%%%%%%%%%%%%%%%%%%%%%%
%%%%%%%%%%%%%%%%%%%%%%%%%%%%%%%%%%%%%%%%%%%%%%%%%%%%%%%%%%%%%%%%%%%%%%%%

\begin{document}
\maketitle
\noindent \textbf{Abstract --- }This work has the intention of finding zeros of approximate functions or roots, was created methods by Matlab to can streamline the accounts
\\
\\
\noindent\keywords{Matlab, Methods, Roots}
\\
\\
\noindent \textbf{Resumo --- }Este trabalho tem o intuito de achar zeros de funções ou raízes aproximadas, foi criados métodos pelo Matlab para pode agilizar as contas
\\
\\
\noindent\palavraschave{Matlab, Métodos, Raízes}

\begin{multicols}{2}

\section{Introdução}
\lipsum[1-5]

\section{Desenvolvimento}
\lipsum[1-2]
\begin{lstlisting}
clc;
clear;
erro=1000;
i=1;
f=[1,0,0,-10];
a=2;
b=3;
fa=polyval(f,a);
fb=polyval(f,b);
x(1,i)=(a+b)/2;
%Pra ver se e continua
if(fa*fb<0) 
fx=polyval(f,x(1,i));
c=0;
%Condicao de parada 
%estabelecida pelo exercicio
while(erro>0.1) 
c=c+1;
%Se f(a)*f(x), 
%entao x pertence(a,x)
if(fa*fx<0)
% a=a;
b=x(1,i);
%Senao x pertence(x,b)
else (fa*fb>0) 
a=x(1,i);
% b=b;
end
i=i+1;
%Formula do metodo 
%de bissecao
x(1,i)=(a+b)/2;
fx=polyval(f,x(1,i));
%calculo do erro
erro =abs( x(1,i)- x(1,i-1));
end
end
disp('Numero de iteracoes:');
disp(c);
disp('ERRO:');
disp(erro);
x
\end{lstlisting}

De acordo com \cite{braga2000redes} é assim assim assado.

Um exemplo de citação usando abntex, segundo \cite{haykin2001redes}, blá blá blá.

Para \cite{barreto2002}, blá.

\begin{equation}
x_{n+1} = X_n - \frac{f(x_n)}{{f}'(x_n)}, n \in \mathbb{N}
\end{equation}

\lipsum[1-2]
\section{Conclusão}
\lipsum[1-1]
\section*{Agradecimentos}
\lipsum[1-1]


\bibliography{referencias}
\end{multicols}
\end{document}