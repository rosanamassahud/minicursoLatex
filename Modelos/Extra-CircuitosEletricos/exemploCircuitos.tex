\documentclass[a4paper,10pt]{article}
%configuração básica para um documento em português
\usepackage[utf8]{inputenc}
\usepackage[brazil]{babel}
\usepackage[T1]{fontenc}
\usepackage{amsmath}
\usepackage{amsfonts}
\usepackage{amssymb}

\usepackage{times}

\usepackage{setspace}
%configuração básica para um documento em português

\usepackage{graphicx}%pacote figuras
\usepackage{graphics}%pacote figuras
\usepackage{caption}%permite colocar legenda nas figuras
\usepackage{adjustbox}%permite utilizar um sistema de escalas quando utilizado o pacote circuitikz

%\usepackage{float}%permite colocar bordas na figura
%\floatstyle{boxed}%permite colocar bordas na figura
%\restylefloat{figure}%permite colocar bordas na figura

\usepackage[siunitx]{circuitikz}%pacote que contém os componentes elétricos: diodos, resistores,etc
\usepackage{tikz}

\usepackage{hyperref}
\usepackage{lipsum}
\begin{document}%início do documento

\section{Introdução}
%Testando o pacote circuitikz - Veja que o componente nao se encontra centralizado
  \begin{circuitikz}
   \draw (0,0) to[resistor] (2,0); %resistor ao longo do eixo x - horizontal à tela do seu computador
  \end{circuitikz}

%Testando o pacote circuitikz - Veja que o componente nao se encontra centralizado
  \begin{circuitikz}
   \draw (0,0) to[resistor] (0,2); %resistor ao longo do eixo x - horizontal à tela do seu computador
  \end{circuitikz}
  %%%%%%%%%%%%%%%%%%%%%%%%%%%%%%%%%%%%%%%%%%%%%%%%%%%%%%%%%%%%%%%%%%%%%%%%
  %% Observação: A cada vez que for utilizado \begin{circuitikz} \end{circuitikz} é criado sistema de referência relativo %%
  %%%%%%%%%%%%%%%%%%%%%%%%%%%%%%%%%%%%%%%%%%%%%%%%%%%%%%%%%%%%%%%%%%%%%%%%  
%Utilizando o comando \begin{center} \end{center} para centralizar o componente
 \begin{center}
  \begin{circuitikz}
   \draw (0,0) to[resistor] (2,0); %resistor
  \end{circuitikz}
 \end{center}
 
%utilizando \begin{figure} \end{}figure - parte 1
 \begin{figure}[!htpb]
 %%%%%%%%%%%%%%%%%%%%%%%%%%%%%%%%%%%%%%%%%%%
 \begin{center}
  \begin{circuitikz}
   \draw (0,0) to[resistor] (2,0);%resistor ao longo do eixo x - horizontal à tela do seu computador desde o ponto (0,0)  até (2,0) 
  \end{circuitikz}
 \end{center}
 %%%%%%%%%%%%%%%%%%%%%%%%%%%%%%%%%%%%%%%%%%%
 \caption{resistor-sem ajuste da escala do componente} % o comando caption para colocar legenda na figura
 \end{figure}
 
 %utilizando \begin{figure} \end{}figure - parte 2
 \begin{figure}[!htpb]%[!htpb] permite colocar a figura na melhor posição
 %%%%%%%%%%%%%%%%%%%%%%%%%%%%%%%%%%%%%%%%%%%
 \begin{center}
 \begin{adjustbox}{scale=0.5}%Início - Ajuste da escala do componente
  \begin{circuitikz}
   \draw (0,0) to[resistor] (2,0); %resistor
  \end{circuitikz}
 \end{adjustbox}%Fim - Ajuste da escala do componente
 \end{center}
 %%%%%%%%%%%%%%%%%%%%%%%%%%%%%%%%%%%%%%%%%%%
 \caption{resistor-utilizando ajuste da escala do componente} % o comando caption para colocar legenda na figura
 \end{figure}
 
 %%%%%%%%%%%%%%%%%%%%%%%%%%Utilizando outro componente%%%%%%%%%%%%%%%%%%%%
 \begin{figure}[!htpb]%[!htpb] permite colocar a figura na melhor posição
 %%%%%%%%%%%%%%%%%%%%%%%%%%%%%%%%%%%%%%%%%%%
 \begin{center}
 \begin{adjustbox}{scale=1.0}%Início - Ajuste da escala do componente
  \begin{circuitikz}
   \draw (0,0) to[capacitor] (2,0); %resistor
  \end{circuitikz}
 \end{adjustbox}%Fim - Ajuste da escala do componente
 \end{center}
 %%%%%%%%%%%%%%%%%%%%%%%%%%%%%%%%%%%%%%%%%%%
 \caption{Capacitor} % o comando caption para colocar legenda na figura
 \end{figure}
 
 \begin{figure}[!htpb]%[!htpb] permite colocar a figura na melhor posição
 %%%%%%%%%%%%%%%%%%%%%%%%%%%%%%%%%%%%%%%%%%%
 \begin{center}
 \begin{adjustbox}{scale=1.0}%Início - Ajuste da escala do componente
  \begin{circuitikz}
   \draw (0,0) to[ammeter] (2,0); %resistor
  \end{circuitikz}
 \end{adjustbox}%Fim - Ajuste da escala do componente
 \end{center}
 %%%%%%%%%%%%%%%%%%%%%%%%%%%%%%%%%%%%%%%%%%%
 \caption{Amperímetro} % o comando caption para colocar legenda na figura
 \end{figure}
 
 \begin{figure}[!htpb]%[!htpb] permite colocar a figura na melhor posição
 %%%%%%%%%%%%%%%%%%%%%%%%%%%%%%%%%%%%%%%%%%%
 \begin{center}
 \begin{adjustbox}{scale=1.0}%Início - Ajuste da escala do componente
  \begin{circuitikz}
   \draw (5,0) to[voltmeter] (9,0); %resistor
  \end{circuitikz}
 \end{adjustbox}%Fim - Ajuste da escala do componente
 \end{center}
 %%%%%%%%%%%%%%%%%%%%%%%%%%%%%%%%%%%%%%%%%%%
 \caption{Voltímetro} % o comando caption para colocar legenda na figura
 \end{figure}
 
 \begin{figure}[!htpb]%[!htpb] permite colocar a figura na melhor posição
 %%%%%%%%%%%%%%%%%%%%%%%%%%%%%%%%%%%%%%%%%%%
 \begin{center}
 \begin{adjustbox}{scale=1.0}%Início - Ajuste da escala do componente
  \begin{circuitikz}
   \draw (5,0) to[sD*] (9,0); %resistor
  \end{circuitikz}
 \end{adjustbox}%Fim - Ajuste da escala do componente
 \end{center}
 %%%%%%%%%%%%%%%%%%%%%%%%%%%%%%%%%%%%%%%%%%%
 \caption{Diodo Stchottky} % o comando caption para colocar legenda na figura
 \end{figure}
 
 \begin{figure}[htpb]%[!htpb] permite colocar a figura na melhor posição
 %%%%%%%%%%%%%%%%%%%%%%%%%%%%%%%%%%%%%%%%%%%
 \begin{center}
 \begin{adjustbox}{scale=1.0}%Início - Ajuste da escala do componente
  \begin{circuitikz}
   \draw (5,0) to[full Zener diode] (9,0); %resistor
  \end{circuitikz}
 \end{adjustbox}%Fim - Ajuste da escala do componente
 \end{center}
 %%%%%%%%%%%%%%%%%%%%%%%%%%%%%%%%%%%%%%%%%%%
 \caption{Diodo Zener} % o comando caption para colocar legenda na figura
 \end{figure}
 
 \begin{figure}[htpb]%[!htpb] permite colocar a figura na melhor posição
 %%%%%%%%%%%%%%%%%%%%%%%%%%%%%%%%%%%%%%%%%%%
 \begin{center}
 \begin{adjustbox}{scale=1.0}%Início - Ajuste da escala do componente
  \begin{circuitikz}
   \draw (5,0) to[led] (9,0); %resistor
  \end{circuitikz}
 \end{adjustbox}%Fim - Ajuste da escala do componente
 \end{center}
 %%%%%%%%%%%%%%%%%%%%%%%%%%%%%%%%%%%%%%%%%%%
 \caption{Led} % o comando caption para colocar legenda na figura
 \end{figure}
 
 \begin{figure}[htpb]%[!htpb] permite colocar a figura na melhor posição
 %%%%%%%%%%%%%%%%%%%%%%%%%%%%%%%%%%%%%%%%%%%
 \begin{center}
 \begin{adjustbox}{scale=1.0}%Início - Ajuste da escala do componente
  \begin{circuitikz}
   \draw (5,0) to[pD*] (9,0); %resistor
  \end{circuitikz}
 \end{adjustbox}%Fim - Ajuste da escala do componente
 \end{center}
 %%%%%%%%%%%%%%%%%%%%%%%%%%%%%%%%%%%%%%%%%%%
 \caption{Fotodiodo} % o comando caption para colocar legenda na figura
 \end{figure}
 
 \begin{figure}[htpb]%[!htpb] permite colocar a figura na melhor posição
 %%%%%%%%%%%%%%%%%%%%%%%%%%%%%%%%%%%%%%%%%%%
 \begin{center}
 \begin{adjustbox}{scale=1.0}%Início - Ajuste da escala do componente
  \begin{circuitikz}
   \draw (0,0) node[npn] (npn) {}
     (npn.base) node[anchor=east] {B}
     (npn.collector) node[anchor=south] {C}
     (npn.emitter) node[anchor=north] {E};
   \draw ($(npn)-(0.18,0)$) circle [radius=18pt];
  \end{circuitikz}
 \end{adjustbox}%Fim - Ajuste da escala do componente
 \end{center}
 %%%%%%%%%%%%%%%%%%%%%%%%%%%%%%%%%%%%%%%%%%%
 \caption{Transistor NPN} % o comando caption para colocar legenda na figura
 \end{figure}
 
 \begin{figure}[htpb]%[!htpb] permite colocar a figura na melhor posição
 %%%%%%%%%%%%%%%%%%%%%%%%%%%%%%%%%%%%%%%%%%%
 \begin{center}
 \begin{adjustbox}{scale=1.0}%Início - Ajuste da escala do componente
  \begin{circuitikz}
   \node[american and port] at (0,1) {};
  \end{circuitikz}
 \end{adjustbox}%Fim - Ajuste da escala do componente
 \end{center}
 %%%%%%%%%%%%%%%%%%%%%%%%%%%%%%%%%%%%%%%%%%%
 \caption{Porta E} % o comando caption para colocar legenda na figura
 \end{figure}
 
\begin{figure}[htpb]%[!htpb] permite colocar a figura na melhor posição
 %%%%%%%%%%%%%%%%%%%%%%%%%%%%%%%%%%%%%%%%%%%
 \begin{center}
 \begin{adjustbox}{scale=1.0}%Início - Ajuste da escala do componente
  \begin{circuitikz}
  \node[transformer core] at (0,1) {};
  \end{circuitikz}
 \end{adjustbox}%Fim - Ajuste da escala do componente
 \end{center}
 %%%%%%%%%%%%%%%%%%%%%%%%%%%%%%%%%%%%%%%%%%%
 \caption{Amplificador operacional} % o comando caption para colocar legenda na figura
 \end{figure}
 
 \begin{figure}[htpb]%[!htpb] permite colocar a figura na melhor posição
 %%%%%%%%%%%%%%%%%%%%%%%%%%%%%%%%%%%%%%%%%%%
 \begin{center}
 \begin{adjustbox}{scale=1.0}%Início - Ajuste da escala do componente
  \begin{circuitikz}
   \node[op amp] at (0,1) {};
   \node[op amp,yscale=-1] at (5,1) {};
  \end{circuitikz}
 \end{adjustbox}%Fim - Ajuste da escala do componente
 \end{center}
 %%%%%%%%%%%%%%%%%%%%%%%%%%%%%%%%%%%%%%%%%%%
 \caption{Amplificador operacional} % o comando caption para colocar legenda na figura
 \end{figure}

\begin{figure}[htpb]%[!htpb] permite colocar a figura na melhor posição
 %%%%%%%%%%%%%%%%%%%%%%%%%%%%%%%%%%%%%%%%%%%
 \begin{center}
 \begin{adjustbox}{scale=1.0}%Início - Ajuste da escala do componente
  \begin{circuitikz}
   \node[currarrow] at (0,1) {};
  \end{circuitikz}
 \end{adjustbox}%Fim - Ajuste da escala do componente
 \end{center}
 %%%%%%%%%%%%%%%%%%%%%%%%%%%%%%%%%%%%%%%%%%%
 \caption{Seta que indica sentido da corrente e/ou tensão} % o comando caption para colocar legenda na figura
 \end{figure} 
 
 \begin{figure}[htpb]%[!htpb] permite colocar a figura na melhor posição
 %%%%%%%%%%%%%%%%%%%%%%%%%%%%%%%%%%%%%%%%%%%
 \begin{center}
 \begin{adjustbox}{scale=1.0}%Início - Ajuste da escala do componente
  \begin{circuitikz}
   \node[circ] at (0,1) {};
  \end{circuitikz}
 \end{adjustbox}%Fim - Ajuste da escala do componente
 \end{center}
 %%%%%%%%%%%%%%%%%%%%%%%%%%%%%%%%%%%%%%%%%%%
 \caption{Nó} % o comando caption para colocar legenda na figura
 \end{figure}
 
 \begin{figure}[!htpb]%[!htpb] permite colocar a figura na melhor posição
 %%%%%%%%%%%%%%%%%%%%%%%%%%%%%%%%%%%%%%%%%%%
 \begin{center}
 \begin{adjustbox}{scale=1.0}%Início - Ajuste da escala do componente
  \begin{circuitikz}
  \draw (0,0) to[R, l=$R_1$] (2,0);
  \end{circuitikz}
 \end{adjustbox}%Fim - Ajuste da escala do componente
 \end{center}
 %%%%%%%%%%%%%%%%%%%%%%%%%%%%%%%%%%%%%%%%%%%
 \caption{Resistor} % o comando caption para colocar legenda na figura
 \end{figure}
 
 \begin{figure}[!htpb]%[!htpb] permite colocar a figura na melhor posição
 %%%%%%%%%%%%%%%%%%%%%%%%%%%%%%%%%%%%%%%%%%%
 \begin{center}
 \begin{adjustbox}{scale=1.0}%Início - Ajuste da escala do componente
  \begin{circuitikz}
  \draw (0,0) to[R, i=$i_1$] (2,0);
  \end{circuitikz}
 \end{adjustbox}%Fim - Ajuste da escala do componente
 \end{center}
 %%%%%%%%%%%%%%%%%%%%%%%%%%%%%%%%%%%%%%%%%%%
 \caption{Resistor - corrente} % o comando caption para colocar legenda na figura
 \end{figure}
 
\begin{figure}[!htpb]%[!htpb] permite colocar a figura na melhor posição
 %%%%%%%%%%%%%%%%%%%%%%%%%%%%%%%%%%%%%%%%%%%
 \begin{center}
 \begin{adjustbox}{scale=1.0}%Início - Ajuste da escala do componente
  \begin{circuitikz}
  \draw (0,0) to[R, l=1<\kilo\ohm>, i=$\SI{1}{\milli\ampere}$, v=1<\volt>] (2,0);
  %\draw (0,0) to[R, v=$v_1$] (2,0);
  %\draw (0,0) to[R=$R_1$,i=$i_1$,v=$v_1$] (2,0);
  \end{circuitikz}
 \end{adjustbox}%Fim - Ajuste da escala do componente
 \end{center}
 %%%%%%%%%%%%%%%%%%%%%%%%%%%%%%%%%%%%%%%%%%%
 \caption{Resistor - tensão} % o comando caption para colocar legenda na figura
 \end{figure}
 
 \section{Circuitos \dots}
 
 \lipsum[1-2]
 \begin{figure}[!h]
 \begin{center}
 	\begin{circuitikz}
 		\draw (0,0)
 		to[V, v=$U_q$] (0,2) %voltage
 		to[short] (2,2)
 		to[R=$R_1$] (2,0) %resistor
 		to[short] (0,0);
 		\draw (2,2)
 		to[short] (4,2)
 		to[L=$L_1$] (4,0)
 		to[short] (2,0);
 	\end{circuitikz}
 \end{center}
 \caption{Primeiro circuito}
 \end{figure}
 
 \lipsum[1-1]
 
 \begin{figure}[!h]
 \begin{center}
 \begin{circuitikz}[american voltages]
\draw
  (0,0) to [short, *-] (6,0)
  to [V, l_=$\mathrm{j}{\omega}_m \underline{\psi}^s_R$] (6,2) 
  to [R, l_=$R_R$] (6,4) 
  to [short, i_=$\underline{i}^s_R$] (5,4) 
  (0,0) to [open, v^>=$\underline{u}^s_s$] (0,4) 
  to [short, *- ,i=$\underline{i}^s_s$] (1,4) 
  to [R, l=$R_s$] (3,4)
  to [L, l=$L_{\sigma}$] (5,4) 
  to [short, i_=$\underline{i}^s_M$] (5,3) 
  to [L, l_=$L_M$] (5,0); 
  \end{circuitikz}
   \end{center}
 \caption{Segundo exemplo - option \it{american voltages}}
 \end{figure}
 
 \lipsum[1-1]
 
  \begin{figure}[!htpb]
 \begin{center}
 \begin{circuitikz}[european voltages]
\draw
  (0,0) to [short, *-] (6,0)
  to [V, l_=$\mathrm{j}{\omega}_m \underline{\psi}^s_R$] (6,2) 
  to [R, l_=$R_R$] (6,4) 
  to [short, i_=$\underline{i}^s_R$] (5,4) 
  (0,0) to [open, v^>=$\underline{u}^s_s$] (0,4) 
  to [short, *- ,i=$\underline{i}^s_s$] (1,4) 
  to [R, l=$R_s$] (3,4)
  to [L, l=$L_{\sigma}$] (5,4) 
  to [short, i_=$\underline{i}^s_M$] (5,3) 
  to [L, l_=$L_M$] (5,0); 
  \end{circuitikz}
   \end{center}
 \caption{O mesmo circuito com option \it{european voltages}}
 \end{figure}
 
 \section*{Referências}
 
 \singlespacing
\noindent Saravia, Edgard Gregory Torres. Curso de Latex. Disponível em: \url{http://www.icea.ufop.br/professores/edgard_gregory/curso_de_latex}. Acesso em: Set/2017.

 \doublespacing

\singlespacing
\noindent ShareLaTeX. CircuiTikz package. Disponível em: \url{https://pt.sharelatex.com/learn/CircuiTikz_package}. Acesso em: Set/2017.

 \doublespacing
 
 \singlespacing
\noindent Redaelli, Massimo A.; Lindner, Stefan; Erhardt, Stefan. CircuiTikZ. (Manual). Mai/2017. Disponível em: \url{http://repositorios.cpai.unb.br/ctan/graphics/pgf/contrib/circuitikz/doc/circuitikzmanual.pdf}. Acesso em: Set/2017.

\end{document}%fim do documento