%Para introduzirmos comentários colocamos um % e escrevemos o que quisermos pois a partir daí o compilador ignora o texto.
%Se quisermos obrigar à mudança de linha, colocamos \\.

%Preâmbulo

%Determina a classe do documento. Classes possíveis (há mais!): letter, article, book, report, slides, beamer.
\documentclass[12pt,a4paper, twocolumn]{article}
%\documentclass[12pt,a5paper]{article}

%% outras possibilidades de tipo de papel:

%%a5paper - A5: 148 mm x 210 mm, that is half of A4, basically
%%b5paper - B5: 250 mm x 176 mm
%%letterpaper - Letter: 215.9 mm x 279.4 mm
%%legalpaper - Legal: 215.9 mm x 355.6 mm

%%%%%%%%%%%%%%%%%%%%%%%%%%%%%%%%%%%%%%%%%%%%%%%%%%%%%%%%%%%%%%%%%%%%%%%%

%Packages(Módulos específicos)

\usepackage{lipsum} %para gerar alguns textos como exemplo
%\usepackage[portuguese]{babel} %para portugues portugal
\usepackage[brazil]{babel} %para portugues brasil
\usepackage[T1]{fontenc} 
\usepackage[utf8]{inputenc}
%\usepackage[latin1]{inputenc}

\usepackage{setspace} %espaçamento entre linhas
\usepackage[top=3cm,left=3cm,right=2cm,bottom=2cm]{geometry} %definir margens
\setlength{\columnsep}{1cm}
%%%%%%%%%%%%%%%%%%%%%%%%%%%%%%%%%%%%%%%%%%%%%%%%%%%%%%%%%%%%%%%%%%%%%%%%
%%%%%%%%%%%%%%%%%%%%%%%%%%%%%%%%%%%%%%%%%%%%%%%%%%%%%%%%%%%%%%%%%%%%%%%%

%Abertura

%Introdução do título.
\title{TÍTULO DO PRIMEIRO DOCUMENTO EM \LaTeX \\
Um exemplo simples}

%Introdução do autor.
\author{Seu Nome Fulano de Tal}

%Definição da data
\date{\today}


%%%%%%%%%%%%%%%%%%%%%%%%%%%%%%%%%%%%%%%%%%%%%%%%%%%%%%%%%%%%%%%%%%%%%%%%
%%%%%%%%%%%%%%%%%%%%%%%%%%%%%%%%%%%%%%%%%%%%%%%%%%%%%%%%%%%%%%%%%%%%%%%%

\begin{document}

%Gerar o título
\maketitle
\newpage
\tableofcontents
\newpage
%\part{Primeira parte}
%\chapter{Meu primeiro texto em \LaTeX}

\section{Primeira Seção}
\doublespacing Vamos escrever uma frase para testar tamanhos {\small pequenos} e {\large grandes}. Tamanhos {\footnotesize muito pequenos} e também {\Large muito grandes}. Texto {\tiny muito muito pequenos}. Os maiores são {\huge esse} e {\Huge esse}.

\singlespacing
\lipsum[1-10]

\section{Título da segunda seção}
\lipsum[1-10]

%\part{Segunda parte}
%\chapter{Meu primeiro texto em \LaTeX}

\section{Primeira Seção}
Vamos escrever uma frase para testar tamanhos {\small pequenos} e {\large grandes}. Tamanhos {\footnotesize muito pequenos} e também {\Large muito grandes}. Texto {\tiny muito muito pequenos}. Os maiores são {\huge esse} e {\Huge esse}.

\lipsum[1-10]

\section{Título da segunda seção}
\lipsum[1-10]

\end{document}